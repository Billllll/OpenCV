\section{Aportes de  Daniel Arnao Rodríguez}
Debido a mi incorporación al proyecto con un mes de retraso, lo primero 
que tuve que hacer fue ponerme al día con el trabajo que había avanzado 
Miguel en el área del reconocimiento de poros, así como en el uso de 
OpenCV, SimpleCV y Python.

\subsection{Preparación}
Antes de abordar los problemas planteados por el proyecto era necesario
familiarizarse con las técnicas que brindan OpenCV y SimpleCV. \\
SimpleCV no dio demasiado trabajo debido a la potencia de la función
\emph{findBlobs}, capaz de detectar por sí sola los poros de la papila. \\
OpenCV, en cambio, proveía una gran cantidad de técnicas útiles para el
trabajo que se iba a realizar. Es por ello que, entre Miguel y yo,
desarrollamos una serie de programas auxiliares que nos permitieran
estudiar, de forma rápida y sencilla, el comportamiento de dichas 
técnicas cuando se aplican con diferentes parámetros y el resultado
de sus combinaciones. Estos programas son los descritos en el capítulo 7.

\subsection{Concreción de los objetivos}
A pesar de mi incorporación tardía al proyecto, los objetivos no estaban
definidos con suficiente nivel de detalle, especialmente los referentes
a parte de medir el espesor de la coroides. Fue más adelante, durante 
una reunión con los oftalmólogos en el \emph{Hospital 12 de Octubre},
cuando tuve la oportunidad de preguntarles directamente lo que esperaban
de nuestro trabajo. De esta forma los objetivos quedaron perfectamente
claros y matizados.

\subsection{Desarrollo}
Durante la fase de desarrollo tomé una posición principalmente teórica,
en el sentido de definir la estructura del algoritmo y los pasos a seguir
para alcanzar los objetivos, mientras que Miguel se ocupó de la
mayor parte de la transformación a código. \\
También me ocupé de realizar las pruebas necesarias para concretar las 
técnicas a emplear, así como de definir los parámetros exactos que 
permitieran marcar la zona a estudiar en cada caso. Esta tarea no fue
sencilla debido a que cada imagen era diferente y la técnica que 
funcionaba con una, fallaba con otras. Con mucho esfuerzo y dándole
muchas vueltas conseguí proponer una solución que resultó ser válida 
para todas ellas.\\
En cuanto a la implementación, mi tarea principal consistió
en asistir a Miguel cuando se atascaba en algún paso. Aún así también
hice aportaciones propias al código, como el cálculo del punto de 
la \emph{fóvea}.\\
La última tarea que desempeñé durante esta fase fue analizar la 
proporción de micras por píxel para obtener la longitud del espesor
de la coroides. Para ello me basé en una pequeña línea que aparecía
en las tomografías, en la zona inferior izquierda de las imágenes.


\subsection{Elaboración de la memoria}
Todos los apartados de la memoria han sido redactados colaborativamente, 
de modo que no puedo adjudicarme la autoría total de ninguno de ellos.\\
La adaptación de todo el texto de la memoria a un lenguaje propio de 
este tipo de trabajos (TFG) ha sido tarea mía.

\subsection{Conclusiones y valoración personal}
Participar en este proyecto ha sido una experiencia que nunca olvidaré. 
He aprendido a desenvolverme en un entorno prácticamente desconocido,
con lenguajes de programación que nunca había usado y nuevas técnicas.\\
Estoy muy satisfecho de haber colaborado en este avance en el mundo
de \emph{Computer Vision} y de ayudar, en la medida de lo posible, a
mejorar los diagnósticos médicos de enfermedades oculares.\\
Sin embargo, no puedo dejar de pensar que aún queda mucho que investigar.
Este solo ha sido mi \emph{granito de arena}.
