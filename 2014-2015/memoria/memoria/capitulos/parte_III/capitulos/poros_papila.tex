\chapter{Medición del área de los poros de la papila}
\section{Problema}
En el curso de la investigación que se está llevando a cabo
actualmente sobre el \emph{glaucoma} y las causas que lo provocan, se
ha detectado una posible relación con la aparición y tamaño de
porosidades en la \emph{papila retiniana}. \\
La \emph{papila retiniana} o \emph{disco óptico} es el punto donde el
nervio óptico entra en el globo ocular. Tiene la forma de un pequeño
disco rosado de aproximadamente $2 \times 1.5$ milímetros y está
situado en la parte posterior del globo ocular. Hasta hace
relativamente poco no se sabía mucho de ella debido a las carencias
tecnológicas. Gracias a las \gls{tco}, se han podido conseguir datos
con cierta precisión, entre los que se encuentra el objeto de este
estudio: unos poros de los que se cree que puedan estar
relacionados con la aparición del \emph{glaucoma}.\\
% Hay que añadir cómo se generan las imágenes

\section{Objetivos}
El objetivo principal de esta parte de la investigación implica
encontrar automáticamente, mediante algoritmos de visión
computerizada, los poros del disco óptico para medirlos y, en la
medida de lo posible, establecer una correlación entre su tamaño, la
cantidad, la posición o la forma y la aparición del \emph{glaucoma},
siendo este procedimiento supervisado por personal médico.
\begin{itemize}
\item Detección de los poros de la papila así como también de la mayor
  cantidad de sus propiedades: área, coordenadas, distribución, etc.
\end{itemize}

\section{Estudio}
Al ser un estudio pionero, lo primero que se procedió a realizar fue
un análisis previo de viabilidad de las \gls{tco} obtenidas. El
principal problema encontrado es la dificultad de enfoque de la papila
que provoca que las imágenes sean totalmente distintas por cada
paciente. Sin embargo, con el enfoque adecuado, si es viable proceder
a estudiar los poros aunque no hay una forma clara de estandarizar el
proceso. Una vez demostrado la viabilidad de las \gls{tco} que produce
la máquina, se procedió a realizar el estudio.\\
\begin{enumerate}
\item Se realiza una búsqueda de los poros mediante la técnica
  \emph{findBlobs\deftecnica{tecnica:blobs}} de \emph{SimpleCV} con
  área máxima de $150$ píxeles.
\end{enumerate}

\section{Algoritmo propuesto}
