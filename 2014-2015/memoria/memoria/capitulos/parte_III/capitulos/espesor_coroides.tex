\chapter{Medición del espesor de la coroides}
\section{Problema}
Los estudios demuestran que hay una relación entre la inflamación de
la \emph{coroides} y la aparición de la \emph{uveítis}. Es necesario,
por tanto, medir el espesor de la coroides para prevenir esta
enfermedad y para seguir su evolución en el tiempo.\\
La \emph{coroides} o \emph{úvea} es una membrana por la que pasan gran
cantidad de vasos sanguíneos, cuya función es mantener la temperatura
constante y nutrir ciertas estructuras del globo ocular. \\
Gracias a las \gls{tco} es posible medir el grosor de la
membrana. Sin embargo, el software de la máquina no realiza la
medición de forma automática, sino que es necesario trazar la línea a
mano. Esto ralentiza el proceso y, además, genera imprecisiones en las
mediciones debidos principalmente a fallos humanos: Si la imagen no
está en posición horizontal es difícil trazar una línea perpendicular,
puede fallar el pulso, etc. Esta línea se debe trazar por el punto
menos profundo de la retina, conocido como \emph{fóvea} y de forma
perpendicular a la \emph{membrana de Bruch}. Una vez trazada la línea,
hay que medirla.


\section{Objetivos}
Los objetivos puestos en común después de sucesivas reuniones fueron:
\begin{itemize}
\item Automatizar lo máximo posible o requerir intervención mínima del
  oftalmólogo en el proceso de marcado de los puntos del espesor a medir de la
  \emph{coroides}.
\item Una vez delimitado el espesor de la \emph{coroides},
  proporcionar su medida en micras lo más fielmente posible.
\end{itemize}
Cumplir estos objetivos propuestos proporcionaría al estudio de la
evolución y control de la \emph{uveítis} una automatización completa
del seguimiento, pudiendo realizar las medidas de todo el historial de
un paciente en pocos segundos. Lo más importante no es el tiempo
ahorrado sino la erradicación de errores de precisión manuales y
resultados distintos de una misma imagen.

\section{Estudio}
El estudio se ha realizado en distintas fases, todas ellas de manera
secuencial y acumulativa. Además de la documentación de cada fase, se
proporcionan las mismas imágenes de guía y apoyo usadas durante el
estudio.  Sin embargo, aunque las fases aquí expuestas se encuentren
en orden secuencial, el orden cronológico en el que se resolvieron
varía, debido a que con cada nueva imagen que se estudiaba surgían
nuevos problemas.
\begin{enumerate}
\item El primer paso consistió en \textbf{extraer} de la imagen
  original \textbf{el área de interés}. Las máquinas \gls{tco}
  siempre proporcionan, además
  de la zona de estudio, el corte usado sobre la estructura ocular. \\
  \textbf{Se desarrolló una biblioteca encargada de automatizar la
    extracción del área de estudio.}
\item El corte programado durante la exploración tiene un grado de
  inclinación sobre la estructura ocular. \textbf{En la mayoría de las
    exploraciones el corte no es horizontal} o ,si se realizan
  exploraciones desde todos los ángulos, sólo uno muestra la estructura
  de manera horizontal. \\
  \textbf{Se desarrolló un algoritmo para la corrección de la
    inclinación} y así rotar la imagen el grado de desviación del
  corte con respecto a la horizontal de la estructura ocular.
\item Corregida la inclinación de la estructura y para poder realizar
  la medición de la \emph{coroides}, \textbf{es necesario} previamente
  \textbf{establecer un punto de referencia, la \emph{fóvea}}. \\
  \textbf{Se desarrolló un algoritmo para la detección del mínimo
    local de la superficie de la retina, la \emph{fóvea}.}
\item Una vez fijada la \emph{fóvea} \textbf{se marcan los dos puntos
    de la \emph{coroides} necesarios para poder medir su espesor}. \\
  \textbf{Se desarrolló un algoritmo para la detección de los puntos
    de espesor de la \emph{coroides}}
\item Finalmente, \textbf{se realiza la medición del espesor de la
    \emph{coroides}}. Este es el proceso más delicado e impreciso de
  todo el estudio porque al aplicar tantas técnicas de procesamiento a
  la imagen de manera consecutiva, hay que dar una medida en micras
  con el mínimo error de precisión posible del espesor de la \emph{coroides}. \\
  \textbf{Se desarrolló un algoritmo para la medición en micras del
    espesor de la \emph{coroides}}
\end{enumerate}


\section{Algoritmo propuesto}
\begin{enumerate}
\item La primera fase consistió en delimitar y extraer el área a
  investigar. Esto es así, porque las máquinas \gls{tco} en una misma
  imagen presentan el corte analizado y su resultado.\\
  Para resolver este problema de forma automática se procedió a
  realizar una biblioteca que realiza el siguiente procedimiento a la
  imagen.
  \begin{enumerate}[label*=\arabic*.]
  \item Convierte la imagen de escala \emph{RGB} a \emph{escala de
      grises}
  \item Duplica la imagen, en una aplica un \emph{threshold Otsu} y en
    otra un \emph{threshold binario} para \emph{binarizar} la imagen.
  \item Se recorre la imagen del \emph{threshold Otsu} con una
    horizontal de derecha a izquierda, desde la esquina superior
    derecha con las coordenadas, 0 como $x$ y ancho de la imagen como
    $y$ encontrar el borde de separación entre las dos imágenes.
  \item Al conocer el  \textbf{punto del borde de separación en la parte
    superior}, queda buscarlo también en la parte inferior de abajo a arriba en la
    imagen del \emph{threshold binario} con las coordenadas, el borde
    de separación como $x$ y la altura de la imagen como $y$.
  \item Con estos puntos ya se puede obtener la zona de la imagen
    original a estudiar. Si se apura y se recorta el negro sobrante de
    la parte inferior evita errores en la detección del segundo punto
    de la \emph{coroides} en algunos casos.
    \begin{enumerate}[label*=\arabic*.]
    \item Para eliminar dicha zona y como la imagen no tiene por qué
      estar en posición horizontal, se procede a buscar dos puntos
      auxiliares: uno empieza con la $x$ igual a $2/4$ de la anchura
      de la imagen, y el otro con la $x$ igual a $3/4$ de la
      anchura. Ambos empiezan con $y$ igual al borde inferior
      provisional hacia arriba, ignorando toda el área negra.
    \item Una vez obtenidos los dos puntos, se procede a obtener el
      punto definitivo para generar el rectángulo que contiene la
      imagen de estudio.
    \item Para calcular dicho punto, el que está situado más cerca del
      borde inferior para no recortar parte de la propia imagen si no
      está horizontal, hay que calcular primero otros dos. Esos puntos
      son la intersección de una recta imaginaria con los puntos
      auxiliares del paso anterior. Se calculan añadiendo a la $y$ de
      cada punto auxiliar la diferencia con respecto a la $y$ del otro
      punto auxiliar.
    \item Obtenidos estos dos puntos, el que tenga mayor $y$, es
      \textbf{el más cercano al borde inferior} y por tanto el
      utilizado como base para obtener el segundo punto del
      rectángulo.
    \item Finalmente, \textbf{se obtiene el rectángulo que contiene a}
      la parte de \textbf{la imagen} que queremos estudiar \textbf{con
        el punto del borde de separación de la parte superior y el
        punto} formado por la $y$ del punto \textbf{más cercano al
        borde inferior} del paso anterior y la anchura de la imagen
      original como la $x$.
    \end{enumerate}
  \end{enumerate}

\end{enumerate}