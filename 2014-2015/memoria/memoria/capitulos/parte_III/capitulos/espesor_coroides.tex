\chapter{Medición del espesor de la coroides}
\section{Problema}
Los estudios demuestran que hay una relación entre la inflamación de
la \emph{coroides} y la aparición de la \emph{uveítis}. Es necesario,
por tanto, medir el espesor de la \emph{coroides} para prevenir esta
enfermedad y para seguir su evolución en el tiempo.\\
La \emph{coroides} o \emph{úvea} es una membrana por la que pasan gran
cantidad de vasos sanguíneos, cuya función es mantener la temperatura
constante y nutrir ciertas estructuras del globo ocular. \\
Gracias a las \gls{tco} es posible medir el grosor de la membrana. Sin
embargo, el software de la máquina no realiza la medición de forma
automática, sino que es necesario trazar la línea a mano. Esto
ralentiza el proceso y, además, genera imprecisiones en las mediciones
debidos principalmente a fallos humanos: Si la imagen no está en
posición horizontal es difícil trazar una línea perpendicular, puede
fallar el pulso, etc. Esta línea se debe trazar por el punto menos
profundo de la retina, conocido como \emph{fóvea} y de forma
perpendicular a la \emph{membrana de Bruch}. Una vez trazada
 la línea,
hay que medirla.


\section{Objetivos}
Los objetivos puestos en común después de sucesivas reuniones fueron:
\begin{itemize}
\item Automatizar lo máximo posible o requerir intervención mínima del
  oftalmólogo en el proceso de marcado de los puntos del espesor a
  medir de la \emph{coroides}.
\item Una vez delimitado el espesor de la \emph{coroides},
  proporcionar su medida en micras lo más fielmente posible.
\end{itemize}
Cumplir estos objetivos propuestos proporcionaría al estudio de la
evolución y control de la \emph{uveítis} una automatización completa
del seguimiento, pudiendo realizar las medidas de todo el historial de
un paciente en pocos segundos. Lo más importante no es el tiempo
ahorrado sino la erradicación de errores de precisión manuales y
resultados distintos de una misma imagen.

\section{Estudio}
El estudio se ha realizado en distintas fases, todas ellas de manera
secuencial y acumulativa. Además de la documentación de cada fase, se
proporcionan las mismas imágenes de guía y apoyo usadas durante el
estudio.  Sin embargo, aunque las fases aquí expuestas se encuentren
en orden secuencial, el orden cronológico en el que se resolvieron
varía, debido a que con cada nueva imagen que se estudiaba surgían
nuevos problemas.
\begin{enumerate}
\item El primer paso consistió en \textbf{extraer} de la imagen
  original \textbf{el área de interés}. Las máquinas \gls{tco} siempre
  proporcionan, además
  de la zona de estudio, el corte usado sobre la estructura ocular. \\
  \textbf{Se desarrolló una biblioteca encargada de automatizar la
    extracción del área de estudio.}
\item El corte programado durante la exploración tiene un grado de
  inclinación sobre la estructura ocular. \textbf{En la mayoría de las
    exploraciones el corte no es horizontal}, si se realizan
  exploraciones desde todos los ángulos, sólo uno muestra la
  estructura
  de manera horizontal. \\
  \textbf{Se desarrolló un algoritmo para la corrección de la
    inclinación} y así rotar la imagen el grado de desviación del
  corte con respecto a la horizontal de la estructura ocular.
\item Corregida la inclinación de la estructura y para poder realizar
  la medición de la \emph{coroides}, \textbf{es necesario} previamente
  \textbf{establecer un punto de referencia, la \emph{fóvea}}. \\
  \textbf{Se desarrolló un algoritmo para la detección del mínimo
    local de la superficie de la retina, la \emph{fóvea}.}
\item Una vez fijada la \emph{fóvea} \textbf{se marcan los dos puntos
    de la \emph{coroides} necesarios para poder medir su espesor}. \\
  \textbf{Se desarrolló un algoritmo para la detección de los puntos
    de espesor de la \emph{coroides}}
\item Finalmente, \textbf{se realiza la medición del espesor de la
    \emph{coroides}}. Este es el proceso más delicado e impreciso de
  todo el estudio porque al aplicar tantas técnicas de procesamiento a
  la imagen de manera consecutiva, hay que dar una medida en micras
  con el mínimo error de precisión posible del espesor de la \emph{coroides}. \\
  \textbf{Se desarrolló un algoritmo para la medición en micras del
    espesor de la \emph{coroides}}
\end{enumerate}

\section{Algoritmo propuesto}
\begin{enumerate}
\item La primera fase consistió en delimitar y extraer el área\deftecnica{tecnica:roi} a
  investigar. Esto es así, porque las máquinas \gls{tco} en una misma
  imagen presentan el corte analizado y su resultado.\\
  Para resolver este problema de forma automática se procedió a
  realizar una biblioteca que realiza el siguiente procedimiento a la
  imagen.
  \begin{enumerate}[label*=\arabic*.]
  \item Convierte la imagen de escala \emph{RGB} a \emph{escala de
      grises\deftecnica{tecnica:cambio-color}}.
  \item Duplica la imagen, en una aplica un \emph{threshold
      Otsu\deftecnica{tecnica:threshold-otsu}} y en otra un
    \emph{threshold binario\deftecnica{tecnica:threshold-binario}} para \emph{binarizar} la imagen.
  \item Se recorre la imagen del \emph{threshold Otsu} con una
    horizontal de derecha a izquierda, desde la esquina superior
    derecha con las coordenadas, $0$ como $x$ y ancho de la imagen
    como $y$ encontrar el borde de separación entre las dos imágenes.
  \item Al conocer el \textbf{punto del borde de separación en la
      parte superior}, queda buscarlo también en la parte inferior de
    abajo a arriba en la imagen del \emph{threshold binario} con las
    coordenadas, el borde de separación como $x$ y la altura de la
    imagen como $y$.
  \item Con estos puntos ya se puede obtener la zona de la imagen
    original a estudiar. \textbf{Si se apura y se recorta el negro
      sobrante de la parte inferior evita errores} en la detección del
    segundo punto de la \emph{coroides} en algunos casos.
    \begin{enumerate}[label*=\arabic*.]
    \item \textbf{Para eliminar dicha zona} y como la imagen no tiene
      por qué estar en posición horizontal, \textbf{se procede a
        buscar dos puntos auxiliares}: uno empieza con la $x$ igual a
      $2/4$ de la anchura de la imagen, y el otro con la $x$ igual a
      $3/4$ de la anchura. Ambos empiezan con $y$ igual al borde
      inferior provisional hacia arriba, ignorando toda el área negra.
    \item \textbf{Una vez obtenidos los dos puntos, se procede a
        obtener el punto definitivo para generar el rectángulo} que
      contiene la imagen de estudio.
    \item \textbf{Para calcular dicho punto}, el que está situado más
      cerca del borde inferior para no recortar parte de la propia
      imagen si no está horizontal, \textbf{hay que calcular primero}
      otros dos. Esos puntos son \textbf{la intersección con el borde
        de sepación y de una recta imaginaria con los puntos
        auxiliares} del paso anterior. Se calculan añadiendo a la $y$
      de cada punto auxiliar la diferencia con respecto a la $y$ del
      otro punto auxiliar.
    \item Obtenidos estos dos puntos, el que tenga mayor $y$, es el
      más cercano al borde inferior y por tanto el utilizado como base
      para obtener el segundo punto del rectángulo.
    \item Finalmente, \textbf{se obtiene el rectángulo que contiene a}
      la parte de \textbf{la imagen} que queremos estudiar \textbf{con
        el punto del borde de separación de la parte superior y el
        punto} formado por la $y$ del punto \textbf{más cercano al
        borde inferior} del paso anterior y la anchura de la imagen
      original como la $x$.
    \end{enumerate}
  \end{enumerate}
\item Una vez delimitada la zona de interés, \textbf{es necesario
    corregir la inclinación del corte de la máquina} con respecto de
  la estructura ocular. \\
  Para esta labor se desarrolló una biblioteca, que aplica el
  siguiente procedimiento a la imagen.
  \begin{enumerate}[label*=\arabic*.]
  \item \textbf{El primer paso mide el ángulo de inclinación} de la
    estructura ocular \textbf{para así poder realizar la rotación que
      la corrija}.  Para ello se busca y escoge un ángulo de
    referencia, aplicando la \emph{Transformada de
      Hough\deftecnica{tecnica:hough}} para rectas. La estructura
    ocular posee una capa con una intensidad muy alta denominada
    \emph{membrana de Bruch} en la que si se aplica dicha
    transformada, se obtiene sobre ella una recta que puede ser usada
    como referencia. Todas las líneas detectadas están almacenadas de
    forma paramétrica en una matriz de pares
    $\left(\rho, \theta \right)$. La primera línea de la matriz que no
    sea vertical ($\theta \neq 0º$) es la recta buscada sobre la
    membrana.
  \item \textbf{Una vez obtenida la $\theta$ de la pendiente a corregir
      respecto la horizontal} $\left( \theta = 90º \right)$ hay que
    calcular la diferencia de inclinación. Para ello se calcula la
    diferencia con la siguiente fórmula:
    \begin{equation*}
      \theta_\text{Corrección} = \theta_{Bruch} - \theta_{horizontal}
    \end{equation*}
    \begin{center}
      siendo $\theta_{horizontal} = 90º$
    \end{center}
  \item \textbf{Finalmente} se procede a \textbf{rotar la imagen} con
    la $\theta_{\text{Corrección}}$, usando como centro de rotación el
    punto central de la imagen.
  \end{enumerate}
\item Teniendo la imagen horizontal para evitar fallos debidos a la
  inclinación, se procedió a desarrollar un algoritmo para
  \textbf{encontrar el punto de referencia, la \emph{fóvea}}.
  \begin{enumerate}[label*=\arabic*.]
  \item \textbf{Se somete} una copia de \textbf{la imagen} (para no
    alterar la original) \textbf{a un \emph{median
        Blur\deftecnica{tecnica:blur-median}}} con un \emph{kernel} de
    tamaño $7$ \textbf{para eliminar el ruido granulado}.
  \item Sobre esta copia transformada, \textbf{se aplica un
      \emph{threshold binario}} con valor umbral de $30$. Con esto se
    pretende binarizar la imagen para facilitar el siguiente paso.
  \item \textbf{Una vez binarizada la imagen, se procede a hacer un
      ``barrido'' vertical}, de arriba hacia abajo y de izquierda a
    derecha. Para reducir el tiempo de computación, se decidió
    recorrer el eje horizontal de la imagen empezando desde un tercio
    de la anchura y acabando en los dos tercios, una vez se aseguró
    que la \emph{fóvea} siempre se encontraba en esa zona. \\
    \textbf{ El objetivo de este barrido es el de encontrar}, en cada
    iteración vertical, el primer punto blanco, que se corresponde con
    la línea en la que está \textbf{la \emph{fóvea}}. Además, mientras
    itera sobre el eje horizontal, busca la posición en el que este
    punto blanco está más abajo.
  \item En más de una ocasión surge un pequeño inconveniente que
    proviene de someter la imagen al \emph{threshold binario}: no hay
    un ``punto más bajo'', sino que aparece una línea horizontal en su
    lugar porque el \emph{threshold binario} transforma las curvas en
    escaleras, con muchos ``puntos más bajos'', haciendo que la
    \emph{fóvea} quedara descentrada. Para resolver esto y tras
    comprobar la simetría de esta línea con respecto a la vertical que
    pasa por la \emph{fóvea}, en lugar de un único punto, se
    calcularon dos: El primero, el punto de la línea más a la
    izquierda; el segundo, el punto de la línea más a la
    derecha. Haciendo la media de dichos
    puntos, se encuentra el punto medio: la \emph{fóvea}. \\
    Nótese que en los casos en que esta línea no se genera y hay un
    único punto ``más bajo'', este algoritmo sigue siendo válido, pues
    el punto más a la derecha y el punto más a la izquierda coinciden,
    y la media de un elemento repetido es el mismo elemento.
  \end{enumerate}
\item \textbf{La última fase se divide en} tres grandes pasos:
  encontrar el punto superior de la \emph{coroides} por el que pasa la
  misma vertical que por la \emph{fóvea}; encontrar el punto inferior
  de la \emph{coroides} por el que pasa la misma vertical; medir la
  distancia entre los dos puntos y mostrar el resultado en micras.
  \begin{enumerate}[label*=\arabic*.]
  \item \textbf{Primero se determina la posición del punto superior de
      la \emph{coroides}} de la siguiente manera:
    \begin{enumerate}[label*=\arabic*.]
    \item Se somete una copia de la imagen horizontal a un
      \emph{threshold binario} con un valor umbral de $179$ y se usa
      un \emph{Canny} para obtener los bordes.
    \item Sobre la imagen resultante se busca en la vertical que pasa
      por la \emph{fóvea}, de abajo hacia arriba y empezando en un
      cuarto de la altura de la imagen (esto evita fallos provocados
      por una zona negra generada durante la corrección de la
      inclinación), el primer punto negro. Este será el punto superior
      de la \emph{coroides}.
    \end{enumerate}
  \item \textbf{Luego se determina la posición del punto inferior de
      la \emph{coroides}} de la siguiente manera:
    \begin{enumerate}[label*=\arabic*.]
    \item Como en el paso anterior, se transforma una copia de la
      imagen para calcular el punto. En este ocasión se utiliza
      primero un \emph{median Blur} con un \emph{kernel} de tamaño
      $13$ para eliminar ruido granulado.
    \item Una vez eliminado el ruido, se utiliza un \emph{threshold
        adaptativo
        gaussiano\deftecnica{tecnica:threshold-adaptativo-gauss}} con
      un valor umbral de $173$ para \emph{binarizar} la imagen.
    \item Usando la función
      \emph{findContours\deftecnica{tecnica:contornos}} se genera un
      \emph{Canny\deftecnica{tecnica:canny}} para encontrar los
      bordes, descartando los bordes más pequeños (ruido no eliminado
      por el \emph{Median Blur}).
    \item Una vez eliminado el ruido y detectados los bordes, se busca
      sobre la vertical que pasa por la \emph{fóvea}, de abajo hacia
      arriba y empezando en los $6/7$ de la altura de la imagen, el
      primer punto blanco, que se corresponde con el borde inferior de
      la \emph{coroides} y señala el segundo punto que se necesita
      para medir su espesor.
    \end{enumerate}
  \item \textbf{Una vez determinados los dos puntos que delimitan la
      coroides se puede determinar la medida de la misma}.
    \begin{enumerate}[label*=\arabic*.]
    \item Como la imagen está horizontal y los dos puntos encontrados
      se encuentran sobre la misma vertical, la distancia entre ellos
      se puede calcular restando los valores de sus coordenadas sobre
      el eje de ordenadas.
    \item Como esta resta devuelve un resultado en \emph{píxeles}, es
      necesario calcular la relación entre micras y píxeles. Para ello
      se ha usado la proporción indicada en la parte inferior
      izquierda de la imagen original.
    \end{enumerate}
  \end{enumerate}
\end{enumerate}