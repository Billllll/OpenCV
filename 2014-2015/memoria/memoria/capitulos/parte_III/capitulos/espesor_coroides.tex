\chapter{Medición del espesor de la coroides}
\section{Problema}
Los estudios demuestran que hay una relación
entre la inflamación de la \emph{coroides}
y la aparición de la \emph{uveítis}. Es 
necesario, por tanto, medir el espesor de
la coroides para prevenir esta enfermedad
y para seguir su evolución en el tiempo.\\
La \emph{coroides} o \emph{úvea} es una 
membrana por la que pasan gran cantidad de 
vasos sanguíneos, cuya función es mantener 
la temperatura constante y nutrir ciertas 
estructuras del globo ocular. \\
Gracias a las \gls{tcoa} es posible medir 
el grosor de la membrana. Sin embargo, 
el software de la máquina no realiza la 
medición de forma automática, sino que es 
necesario trazar la línea a mano. Esto 
ralentiza el proceso y, además, 
genera imprecisiones en las mediciones 
debidos principalmente a fallos humanos: 
Si la imagen no está en posición horizontal
es difícil trazar una línea perpendicular, 
puede fallar el pulso, etc. Esta línea se 
debe trazar por el punto menos profundo 
 de la retina, conocido como 
\emph{fóvea} y de forma perpendicular a la  
\emph{membrana de Bruch}.
%Falta completar


\section{Objetivos}
Los objetivos puestos en común después de sucesivas reuniones fueron:
\begin{itemize}
\item Automatizar lo máximo posible o requerir intervención mínima del
  oftalmólogo en el proceso de marcado de los puntos del espesor de la
  \emph{coroides} a medir.
\item Una vez delimitado el espesor de la \emph{coroides}, proporcionar su
  medida en micras lo más fielmente posible.
\end{itemize}
Cumplir estos objetivos propuestos proporcionaría al estudio de la
evolución y control de la \emph{uveitis} una automatización completa
de este seguimiento pudiendo realizar las medidas de todo el historial
de un paciente en pocos segundos. Lo más importante no es el tiempo
ahorrado sino la erradicación de errores de precisión manuales y
resultados distintos de una misma imagen.

\section{Estudio}
El estudio se ha realizado en distintas fases todas ellas de manera
secuencial y acumulativa. Además de la documentación de cada fase, se
proporcionan las mismas imágenes de guía y apoyo usadas durante el estudio.
\section{Algoritmo propuesto}
