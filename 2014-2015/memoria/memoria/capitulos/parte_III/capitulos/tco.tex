\chapter{Tomografías de Coherencia Óptica}
Toda la información de este capítulo ha sido cuidadosamente
seleccionada y adaptada de \emph{Avances en la exploración de la
  retina: OCT:\@ Qué es y para qué sirve}~\citep*{oct-bib}

\section{Imágenes TCO}
Las \gls{tco} son un tipo de imágenes ampliamente utilizadas en el
ámbito hospitalario por los oftalmólogos para estudiar, explorar y
monitorizar las estructuras oculares de los pacientes para el
diagnóstico de enfermedades, especialmente, de difícil identificación
oftalmológica.

\section{Máquina de TCO}
Una máquina de \gls{tco} se utiliza para la exploración en tiempo
real, sin ser invasiva ni dolorosa, para obtener con resolución
micrométrica imágenes de estructuras de tejidos vivos. Con el paso del
tiempo se ha hecho insustituible en el diagnóstico y seguimiento de
numerosas patologías oculares, así como también por la mejora de su
rapidez y precisión.

\subsection{Propiedades ópticas de los tejidos}
La máquina de \gls{tco} es capaz de medir y representar los distintos
tipos de tejidos en escala de grises o colores según su reflectividad:
\begin{description}
\item[Reflectividad alta:] sangre, exudados lipídicos, epitelio
  pigmentario, coriocapilar o capas de fibras nerviosas
  perpendiculares al haz de luz. Representados con colores cálidos
  como rojo y blanco.
\item[Reflectividad media:] capas retinianas de la limitante interna a
  la plexiforme externa. Representadas con colores verdes y amarillos
\item[Reflectividad baja:] zonas de edema, vítreo y contenido seroso,
  dispuestos en paralelo al haz de luz. Representados con colores
  fríos como azul y negro.
\end{description}

\subsection{Base teórica}
La máquina de \gls{tco} usa una técnica análoga a las máquina de
ecografías pero sustituyendo los ultrasonidos por luz. La máquina mide
la latencia o retardo e intensidad de la onda reflejada tras incidir
en un tejido. La ventaja de una máquina de \gls{tco} frente a una de
ecografías es la velocidad de la luz, muy superior a la velocidad de
los ultrasonidos, del orden de \emph{$10^{-15}$ fentosegundos},
requiriendo un sistema de medición indirecto.\\
\emph{El interferómetro de Michelson} es la principal base del
funcionamiento de estas máquinas. Consiste en dirigir la radiación a
un divisor de haces que a su vez los divide en dos. Un haz se dirige
por un medio conocido y el otro por el medio a estudiar. Tras esto,
ambos haces se reflejan para hacerlos coincidir en un mismo punto,
usando el haz que atraviesa el medio conocido como patrón para
registrar la interferencia entre ambos y medir la intensidad y el
retardo del haz que recorrió el medio desconocido.

\subsection{Historia}\label{tco:historia}
Estas primeras máquinas operaban sobre el \emph{dominio temporal}. Es
decir, el haz se reflejaba sobre un espejo que se movía para para
recorrer todo el espacio a estudiar. La velocidad de traslación del
espejo se convertía en el factor crítico y limitante de la velocidad
de
captura. En los mejores casos se alcanzaban los \emph{400 barridos por segundo}.\\
La velocidad de barrido es crítica principalmente por dos motivos:
\begin{description}
\item [Disminución del tiempo de estudio] de la estructura
  ocular. Esto reduce la irritación y la sequedad del ojo del paciente
  que, al tenerlo abierto durante todo el proceso, hacen que se
  reduzca la calidad del resultado obtenido.
\item [La resolución y calidad] son directamente proporcionales número
  de muestras realizadas. Es decir, cuantos más barridos, mejores
  resultados pero aumentar el número de barridos también aumenta el
  tiempo de exploración, por lo que es necesario buscar un equilibrio
  entre la resolución y la duración de la exploración.
\end{description}
Con el paso del tiempo y el perfeccionamiento de la técnica, se hizo
posible fijar el espejo y aumentar así los barridos por segundo
mediante el uso del \emph{dominio espectral o de Fourier}. Esta
técnica se sustenta en una serie de dispositivos que analizan el haz a
comparar con el de referencia. Con el espejo fijo en un punto y los
nuevos dispositivos, se consiguió hacer los barridos hasta 700 veces
más rápido, alcanzando los \emph{25.000 barridos por segundo}.\\
Actualmente, las máquinas trabajan con una variante del \emph{dominio
  espectral}: las llamadas de tipo \emph{swept-source}. Estas utilizan
el dominio de la frecuencia codificada en el tiempo (\emph{Time
  encoded frequency domain}). Para ello, un láser sintonizable emite
el haz con una longitud de onda fija que muestrea secuencialmente el
medio con una serie de longitudes de onda individuales que
posteriormente descodifica. Así, alcanza una velocidad de
\emph{100.000 muestreos por segundo}. Además, proporciona imágenes muy
precisas y es capaz de explorar a todos los niveles de profundidad,
por lo que es capaz de mostrar simultáneamente (en la misma
exploración) imágenes de gran calidad tanto de estructuras
superficiales como el vítreo, como de las más profundas, como la
\emph{\gls{coroides}} e incluso la \emph{\gls{esclera}}.

\subsection{Tipo, resolución y densidad de muestreo}\label{tco:trd}
\subsubsection{Tipo}
Para la reconstrucción de la imagen de un tejido, la máquina de
\gls{tco} explora el medio con un haz de luz en un punto o
\emph{\textbf{A-scan}}. La repetición de puntos sobre una misma línea
genera un plano denominado \emph{\textbf{B-scan}}. La repetición de
estos planos construye un cubo.
\subsubsection{Resolución}
La resolución de muestreo se define como la distancia mínima entre dos
puntos próximos distintos. Permite distinguir ambos puntos como
diferentes.
\begin{itemize}
\item Resolución \textbf{axial}: limitada por la longitud de
  coherencia de la onda.
\item Resolución \textbf{transversal}: dependiente de la anchura del
  haz de luz incidente.
\end{itemize}
\subsubsection{Densidad}
La densidad de muestreo se define como la cantidad de
\emph{\textbf{A-scan}} por unidad de volumen de tejido. Es
directamente proporcional al tiempo de exploración.

\subsection{Número de muestreos en el mismo punto o \emph{frame
    averaging}}\label{tco:frame}
Para incrementar la calidad de una imagen es recomendable aumentar el
número de muestreos puesto que las máquinas de \gls{tco} usan luz
parcialmente coherente y las reflexiones adyacentes producen
interferencias o ruido en el tejido. Este ruido es siempre distinto en
cada exploración por lo que, al repetir el proceso en cada punto del
tejido y hacer su promedio para reconstruir la imagen, el ruido
desaparece. Esto hace posible distinguir capas con bajo contraste o
muy finas.\\
Aunque el aumento de muestreos a priori puede parecer una buena
técnica para mejorar la calidad, requiere incrementar
considerablemente el tiempo de exploración produciendo en el ojo del
paciente sequedad e irritación. Esto no sólo disminuye la calidad sino
que incluso puede llegar a impedir finalizar correctamente la
exploración. Por lo que hay que buscar un equilibrio entre número de
muestreos y tiempo de exploración.\\
La estimación de calidad-ruido se basa en la recomendación de
\emph{\textbf{100 muestreos de tipo B-scan}}, explicados
anteriormente.

\subsection{Tipos de exploración}\label{tco:exploracion}
Lo habitual es la \emph{exploración lineal} porque son las más rápidas
de realizar y tolerables hasta para los ojos con
fatiga, habitual en las personas a partir de cierta edad.\\
La reconstrucción en 3D de estructuras oculares mediante cubos
proporciona mucha más información y permite seleccionar
individualmente cada corte o punto que se quiera
visualizar. Consecuentemente, alarga el proceso siendo inviable de
realizar a pacientes poco colaboradores o con ojos con tendencia a
secarse.\\
Es importante saber que aunque las zonas a explorar están
estandarizadas, cada máquina de \gls{tco} tiene su propia estrategia.

\subsection{Software de análisis de la información}
Un aspecto crítico de las máquinas de \gls{tco} es el software que
incorporan mediante algoritmos de segmentación para medir las
dimensiones de las estructuras previamente exploradas. Teniendo en
cuenta que al tener cada máquina su propia estrategia de exploración,
un mismo tejido en máquinas de \gls{tco} distintas proporciona medidas
diferentes.

\subsection{Seguimiento}\label{tco:seguimiento}
Las máquinas de \gls{tco} más modernas son capaces de reconocer la
zona que está siendo explorada a partir de características distintivas
como los bordes de la papila o vasos sanguíneos. Permite asegurar así
con exploraciones sucesivas a lo largo del tiempo de una misma zona
que los cambios detectados en un punto son reales y no artefactos de
puntos próximos y distintos.\\
Este seguimiento permite, además, medir la distancia entre los dos
mismos puntos en exploraciones sucesivas,lo cual permite cuantificar
los cambios que puedan producirse.

\subsection{Otras características}
La mayoría de las máquinas de \gls{tco} tienen por objetivo el
segmento anterior o el posterior, pero no los dos simultáneamente, ya
que las longitudes de onda idóneas para cada uno son diferentes.\\
Sin embargo, si se aleja la máquina del ojo, las ondas dirigidas al
segmento posterior también pueden usarse para explorar el segmento
anterior.

\subsection{Informes}
Las máquina de \gls{tco} emiten informes de todas las exploraciones de
uno o ambos ojos con múltiples parámetros. \emph{Informes de análisis}
sobre una única exploración o sobre diversas marcando los cambios. Con
algunas enfermedades además de un informe de análisis, se puede
generar un \emph{informe de evolución} acerca de la posible variación
en el tiempo del grosor de algunas capas.\\
Todos los informes han de ser valorados después de la exploración y
antes de ser tenidos en cuenta, de forma que se verifique la
corrección de sus datos, y que la exploración se ha realizado
correctamente y es fiable.

\section{Aplicaciones clínicas}
\begin{itemize}
\item Medición de grosores.
\item Visualización de alteraciones estructurales.
\item Medición de ángulos.
\end{itemize}

\section{Modelo utilizado}
El modelo de máquina de \gls{tco} utilizada durante todo este trabajo
es la \emph{Spectralis, Heidelberg Engeneering, GmgH}, es de
\emph{dominio espectral} o \emph{Fourier}
(ver\emph{~\ref{tco:historia}~\nameref{tco:historia}}) y almacena las
imágenes en formato \emph{jpeg},
\emph{tif} y \emph{bmp}.\\
Dispone un \emph{láser confocal} de diodo superluminiscente con una
longitud de onda de $870nm$, permite así, explorar medios
relativamente opacos. Capacidad necesaria para pacientes con cierto
grado de catarata.\\
La \gls{tco} se puede realizar con diversos patrones lineales y en
forma de cubos tridimensionales
(ver\emph{~\ref{tco:exploracion}~\nameref{tco:exploracion}}) con una
separación mínima entre cortes de $11\;micras$. El número de
repeticiones o \emph{frames} de cada
barrido oscila entre $1$ y $100$ según se configure manualmente\emph{\ref{tco:frame}~\nameref{tco:frame}}.\\
Una característica muy importante de este equipo es el seguimiento de
la imagen para evitar el desalineamiento durante la exploración a
causa de los movimientos oculares
(ver~\emph{\ref{tco:seguimiento}~\nameref{tco:seguimiento}}). Se
compone de dos haces de láser, uno de los cuales realiza el registro
de la imagen mientras que el otro actúa como guía de la \gls{tco} para
asegurar que las tomografías correspondan a la imagen del fondo de
ojo, y por
tanto, la zona seleccionada para la exploración.\\
El equipo incorpora los protocolos de exploración más habituales para
\emph{mácula} y \emph{\gls{glaucoma}} pero las características
concretas de exploración (patrón, resolución, \emph{frames})
(ver\emph{~\ref{tco:trd}~\nameref{tco:trd}}), se pueden seleccionar
manualmente, y una vez diseñada la configuración deseada, almacenarse
y seleccionarse para que todos los pacientes de un mismo estudio sean
explorados con el mismo protocolo, y por tanto, poder comparar los
resultados.

\subsection{Origen de las imágenes estudiadas}