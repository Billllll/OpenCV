\chapter{Plan de trabajo}
La estructura de la memoria para mayor claridad se ha dividido en
cinco partes:
\begin{description}
\item[Parte I, Antecedentes:] esta primera parte está escrita a modo
  de introducción al trabajo, los problemas y objetivos planteados.
\item[Parte II, Visión computerizada:] esta segunda parte se ha
  dedicado a las bibliotecas y técnicas de visión computerizada
  aplicadas en la investigación. Se ha querido profundizar todo lo
  posible con imágenes de ejemplo usadas durante el estudio después de
  cada explicación.
\item[Parte III, Investigación:] la tercera parte es la parte contiene
  todo lo referente al trabajo de estudio e investigación realizado.
\item[Parte IV, Propuesta software:] esta parte está totalmente
  dedicada al código y los algoritmos propuestos para resolver los
  problemas planteados y alcanzar los objetivos descritos en la
  primera parte.
\item[Parte V, Conclusiones:] finalmente y a modo de cierre esta
  quinta y última parte hace de cierre. Contiene las conclusiones a la
  investigación, y un capítulo dedicado al futuro. A todas las ideas,
  propuestas, problemas y objetivos descubiertos a lo largo de todo el
  trabajo y que por falta de tiempo y conocimientos no han podido ser
  abarcados pero sin duda servirán muy pronto como base a nuevas
  investigaciones y trabajos.
\end{description}