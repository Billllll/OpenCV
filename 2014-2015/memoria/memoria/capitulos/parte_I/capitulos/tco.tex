\chapter{Tomografías de Coherencia Óptica}
Toda la información de este capítulo ha sido cuidadosamente
seleccionada y adaptada de \emph{Avances en la exploración de la
  retina: OCT:\@ Qué es y para qué sirve}\cite{oct-bib}

\section{Imágenes \glsentrytext{tcog}}
Las \gls{tcoa} son un tipo de imágenes ampliamente utilizadas en el
ámbito hospitalario por los oftalmólogos para estudiar, explorar y monitorizar las
estructuras oculares de los pacientes para el diagnóstico de
enfermedades, especialmente, de difícil identificación oftalmológica.

\section{Máquina \glsentrytext{tcog}}
Una máquina \gls{tcoa} se utiliza para la exploración en tiempo real, sin ser
invasiva ni dolorosa, para obtener con resolución micrométrica imágenes
de estructuras de tejidos vivos. Con el paso del tiempo se ha hecho
insustituible en el diagnóstico y control de numerosas patologías
oculares, así como también la mejora de su rapidez y precisión.

\subsection{Propiedades ópticas de los tejidos}
La máquina \gls{tcoa} es capaz de medir y representar los distintos
tipos de tejidos en escala de grises o colores según su reflectividad:
\begin{description}
\item[Reflectividad alta:] sangre, exudados lipídicos, epitelio
  pigmentario, coriocapilar o capas de fibras nerviosas
  perpendiculares al haz de luz. Representados con colores cálidos
  como rojo y blanco.
\item[Reflectividad media:] capas retinianas de la limitante interna a
  la plexiforme externa. Representadas con colores verdes y amarillos
\item[Reflectividad baja:] zonas de edema, vítreo y contenido seroso,
  dispuestos en paralelo al haz de luz. Representados con colores
  fríos como azul y negro.
\end{description}

\subsection{Base teórica}
La máquina \gls{tcoa} usa una técnica análoga a las máquina de
ecografías pero sustituyendo los ultrasonidos por luz. La máquina mide
la latencia o retardo e intensidad de la onda reflejada tras incidir
en un tejido. La ventaja de una máquina \gls{tcoa} frente a una de
ecografías es la velocidad de la luz, muy superior a la velocidad de
los ultrasonidos, del orden de \emph{$10^{-15}$ fentosegundos}, requiriendo un sistema de medición indirecto.\\
\emph{El interferómetro de Michelson} es la principal base del
funcionamiento de estas máquinas. Consiste en dirigir la radiación a
un divisor de haces que a su vez los divide en dos. Un haz se dirige
por un medio conocido mientras que el otro por el medio a
estudiar. Tras esto, ambos haces se reflejan para hacerlos coincidir
en un mismo punto, usándolo como patrón para registrar la interferencia entre
ambos y medir la intensidad y el retardo del haz que recorrió el medio
desconocido.

\subsection{Historia}
Estas primeras máquinas operaban sobre el \emph{dominio temporal}, es decir,
el espejo recorría toda la superficie del medio a estudiar para así
reflejar el haz de referencia. La velocidad de traslación del espejo
se convertía en el factor crítico y limitante de la velocidad de
captura, en los mejores casos, \emph{400 barridos por segundo}.\\
La velocidad de barrido es crítica principalmente por dos motivos:
\begin{description}
\item [Disminución del tiempo de estudio] de la estructura ocular. Esto
  que reduce la irritación y la sequedad del ojo del paciente que, al
  tenerlo abierto durante todo el proceso, hacen que se reduzca la
  calidad del resultado obtenido.
\item [La resolución y calidad] son directamente proporcionales
  número de muestras realizadas. Es decir, cuantos más barridos,
  mejores resultados.
\end{description}
Con el paso del tiempo y el perfeccionamiento de la técnica, se hizo
posible fijar el espejo y aumentar así los barridos por segundo
mediante el uso del \emph{dominio espectral o de Fourier}. Esta
técnica se sustenta en una serie de dispositivos que analizan el haz a
comparar con el de referencia. Con el espejo fijo en un punto y los
nuevos dispositivos, se consiguió hacer los barridos hasta 700 veces
más rápido, alcanzando los \emph{25.000 barridos por segundo}.\\
Actualmente, las máquinas más recientes trabajan con una variante del
\emph{dominio espectral}: las llamadas de tipo
\emph{swept-source}. Estas utilizan el dominio de la frecuencia codificada
en el tiempo (\emph{Time encoded frequency domain}). Para ello, un
láser sintonizable emite el haz con una longitud de onda fija que
muestrea secuencialmente el medio con una serie de longitudes de onda
individuales que posteriormente descodifica. Así, alcanza una velocidad de
\emph{100.000 muestreos por segundo}. Además, proporciona imágenes muy
precisas y con mayor profundidad, mostrando tejidos que anteriormente
había que enfocar en niveles más profundos, como la \textbf{coroides}, o
el vítreo.

\subsection{Tipo, resolución y densidad de muestreo}
\subsubsection{Tipo}
Para la reconstrucción de la imagen de un tejido, la máquina
\gls{tcoa} explora el medio con un haz de luz en un punto o
\emph{\textbf{A-scan}}. La repetición de puntos sobre una misma línea
genera un plano denominado \emph{\textbf{B-scan}}. La repetición de
estos planos construye un cubo.
\subsubsection{Resolución}
La resolución de muestreo se define como la distancia mínima entre dos
puntos próximos distintos. Permite distinguir ambos puntos como
diferentes.
\begin{itemize}
\item Resolución \textbf{axial}: limitada por la longitud de
  coherencia de la onda.
\item Resolución \textbf{transversal}: dependiente de la anchura del
  haz de luz incidente.
\end{itemize}
\subsubsection{Densidad}
La densidad de muestreo se define como la cantidad de
\emph{\textbf{A-scan}} por unidad de volumen de tejido. Es
directamente proporcional al tiempo de exploración.

\subsection{Número de muestreos en el mismo punto o \emph{frame
    averaging}}
Para incrementar la calidad de una imagen es recomendable aumentar el
número de muestreos puesto que las máquinas \gls{tcoa} usan luz
parcialmente coherente y las reflexiones
adyacentes producen interferencias o ruido en el tejido. Este ruido es
siempre distinto en cada exploración por lo que, al repetir el proceso
en cada punto del tejido y hacer su promedio para reconstruir la imagen, el
ruido desaparece. Esto hace posible distinguir capas con bajo contraste o
muy finas.\\
Aunque el aumento de muestreos a priori puede parecer una buena
técnica para mejorar la calidad, requiere incrementar
considerablemente el tiempo de exploración produciendo en el ojo del
paciente sequedad e irritación. Esto no sólo disminuye la calidad sino
que incluso puede llegar a impedir finalizar correctamente la
exploración. Por lo que hay que buscar un equilibrio entre número de
muestreos y tiempo de exploración.\\
La estimación de calidad-ruido se basa en la recomendación de
\emph{\textbf{100 muestreos de tipo B-scan}} explicados anteriormente.

\subsection{Tipos de exploración}
Lo habitual es la \emph{exploración lineal} porque son las más rápidas
de realizar y tolerables hasta para los ojos con
fatiga, habitual en las personas a partir de cierta edad.\\
La reconstrucción en 3D de estructuras oculares mediante cubos
proporciona mucha más información y permite seleccionar
individualmente cada corte o punto que se quiera
visualizar. Consecuentemente, alarga el proceso siendo inviable de
realizar a pacientes poco colaboradores o con ojos con tendencia a
secarse.\\
Es importante saber que aunque las zonas ha explorar están estandarizadas, cada máquina
\gls{tcoa} tiene su propia estrategia.

\subsection{Software de análisis de la información}
Un aspecto crítico de las máquinas \gls{tcoa} es el software que
incorporan mediante algoritmos de segmentación para medir las
dimensiones de las estructuras previamente exploradas. Teniendo en
cuenta que al tener cada máquina su propia estrategia de exploración,
un mismo tejido en máquinas \gls{tcoa} distintas muy seguramente proporcione
medidas diferentes.

\subsection{Seguimiento}
Las máquinas \gls{tcoa} más modernas son capaces de reconocer la zona
que está siendo explorada a partir de características distintivas como
los bordes de la papila o vasos sanguíneos. Permite asegurar así con
exploraciones sucesivas a lo largo del tiempo de una misma zona que
los cambios detectados en un punto son reales y no artefactos de
puntos próximos y distintos.\\
Este seguimiento permite además medir la distancia en micras de cada
punto con respecto al punto de una exploración anterior.

\subsection{Otras características}
La mayoría de las máquinas \gls{tcoa} tienen por objetivo el segmento
anterior o el posterior, pero no los dos simultáneamente, ya que
las longitudes de onda idóneas para cada uno son diferentes.\\
Sin embargo, si se aleja la máquina del ojo, las ondas dirigidas 
al segmento posterior también pueden usarse para explorar el segmento anterior.

\subsection{Informes}
Las máquinas \gls{tcoa} emiten informes de todas las exploraciones de
uno o ambos ojos con múltiples parámetros. \emph{Informes de análisis}
sobre una única exploración o sobre diversas marcando los cambios. Con
algunas enfermedades además de un informe de análisis, se puede
generar un \emph{informe de evolución} acerca del grosor de algunas
capas.\\
Todos los informes para saber que sus datos son los correctos tienen
que tenerse en cuenta sólo después de haber hecho una valoración de la
exploración

\section{Aplicaciones clínicas}
\begin{itemize}
\item Medición de grosores.
\item Visualización de alteraciones estructurales.
\item Medición de ángulos.
\end{itemize}