\chapter{Estado del arte}
Casi lo primero que se tuvo en cuenta al elaborar un trabajo de estas
características, fue hacer un análisis del estado del
arte.\\
Un estudio del estado del arte consiste en la evaluación del estado de
evolución de las técnicas y tecnologías a investigar. Los resultados
de este proceso fueron muy vagos y escasos. Los oftalmólogos recuerdan
de algún estudio en los últimos años acerca de obtener mayor partido a
las imágenes de \gls{tco} mediante algún tipo de procesado de la
estructura ocular de manera externa e independiente de la máquina \gls{tco}.\\
Gracias a nuestro tutor Carlos por ponernos en conocimiento una
noticia de la \gls{ECMI}, descubrimos que en el \emph{Centro para las
  Matemática de la Universidad de Coimbra (Portugal)} en un esfuerzo
interdisciplinar en el campo de la \emph{ingeniería biomédica} aplican
las mismas técnicas e ideas de procesamiento (lógicamente a un nivel
de recursos en todos los ámbitos muy superior a los abarcados aquí) a
las \gls{tco} que dieron lugar a este proyecto. \\
Sin embargo, los objetivos son bien distintos. El objetivo que
persiguen es tan ambicioso como genérico. Quieren desarrollar un
algoritmo que incorpore las ecuaciones de \emph{Maxwell} dependientes
del tiempo a las longitudes de onda para así solucionar el problema de
dispersión para cada una de las capas de la retina de los haces de luz
emitidos por la máquina \gls{tco}. Si bien, supondría un grandísimo
avance en cuanto calidad obtenida que podría dar lugar a una mejora de
precisión y abrir el estudio a nuevas estructuras oculares
difícilmente visibles a todos las investigaciones que utilicen
máquinas \gls{tco}.\\
Como ejemplo de aplicación real, se centran en el \emph{edema macular
  diabético}, una complicación de los pacientes de diabetes la cual es
la primera causa de ceguera entre lo pacientes.
