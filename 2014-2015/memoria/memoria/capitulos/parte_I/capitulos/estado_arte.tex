\chapter{Estado del arte}\label{ch:estado_arte}
Uno de los primeros puntos que se tuvo en cuenta al elaborar un
trabajo de estas características, fue hacer un análisis del estado del
arte.\\
Un estudio del estado del arte consiste en la evaluación del estado de
evolución de las técnicas y tecnologías a investigar. Los oftalmólogos
y el tutor nos proporcionaron numerosos artículos y estudios de los
últimos años acerca de cómo obtener mayor partido a las imágenes de
\gls{tco} mediante algún tipo de procesado de la estructura ocular de
manera externa e independiente de la máquina \gls{tco}. Más
concretamente, sobre \emph{Visión computerizada} aplicada al espesor
de la \gls{coroides}\\
Cabe mencionar algunos artículos anteriores a este trabajo para
situarse en contexto y sustentar la motivación a realizarlo.
\begin{itemize}
\item \emph{\citep*[Automated choroidal segmentation of 1060 nm OCT in
    healthy and pathologic eyes using a statistical
    model]{kajic2012automated}}
\item \emph{\citep*[Automatic segmentation of the choroid in enhanced
    depth imaging optical coherence tomography
    images]{tian2013automatic}}
\item \emph{\citep*[Automatic segmentation of choroidal thickness in
    optical coherence tomography]{alonso2013automatic}}
\item \emph{\citep*[Segmentation of choroidal boundary in enhanced
    depth imaging OCTs using a multiresolution texture based modeling
    in graph cuts]{danesh2014segmentation}}
\end{itemize}
Y creados durante el mismo periodo de tiempo que este trabajo.
\begin{itemize}
\item \emph{\citep*[Evaluation of choroidal thickness via enhanced
    depth-imaging optical coherence tomography in patients with
    systemic hypertension]{gok2015evaluation}}
\item \emph{\citep*[Optical modelling of the human
    retina]{ara2015optical}}
\end{itemize}
Con estos artículos, se refleja el gran interés y esfuerzo
interdisciplinar en el campo de la \emph{ingeniería biomédica} para la
aplicación de la \emph{Visión computerizada} en máquinas
\emph{\gls{tco}} y sus imágenes resultantes.\\
Con el estudio del estado del arte se presenta la base a establecer
para sustentar y justificar el inicio y elaboración de este trabajo o
similares.
