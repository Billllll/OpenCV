\chapter{Introducción}
\section{Problema}
Este proyecto nace de la necesidad de mejorar y automatizar el estudio 
de imágenes en el campo de la medicina, con el fin de aumentar la precisión 
del diagnóstico y agilizarlo. En ocasiones, los aparatos que realizan 
estas imágenes carecen de características necesarias para los médicos 
o de facilidades para realizar un seguimiento de la evolución de un paciente 
en base a su historial. \\
Al tratarse de un problema de gran envergadura y que abarca demasiado
terreno, este proyecto se ha centrado en el estudio de imágenes de \gls{tco}
para realizar el seguimiento de dos enfermedades: la \emph{\gls{uveitis}} y
el \emph{\gls{glaucoma}}. Para ello se ha contado con la colaboración de
oftalmólogos del \emph{Hospital Universitario 12 de Octubre}.

\section{Objetivos}
Aunque el problema a primera vista es demasiado amplio,
los objetivos están muy claros para los oftalmólogos. Tanto es así, 
que no tuvieron problema en indicar a mano sobre una imagen los datos que 
necesitan de las \gls{tco}. Así se definieron los siguientes objetivos:
\begin{itemize}
\item Obtener imágenes tan fieles como sea posible a las marcadas
  manualmente.
\item Conseguir automatizar dicho proceso o, al menos,
  facilitarlo para poder realizarlo en el menor número de pasos posible.
\item Proporcionar los conocimientos obtenidos de los programas, datos y
  medidas para mejorar el entendimiento tanto de las \gls{tco}
  como de las enfermedades estudiadas.
\item Romper con la dependencia del software de la máquina \gls{tco}
  imposible de estudiar, mejorar y adaptar por derechos de autor y
  patentes a los problemas planteados por los oftalmólogos en sus
  estudios.
\item Aprender a desenvolvernos en un entorno ajeno a la informática
  con una gran cantidad de técnicas desconocidas, tratando de alcanzar
  resultados útiles y aplicables en el mundo de la medicina.
\item Comprender el problema en su totalidad, los objetivos y todas
  las soluciones propuestas. Así, en un futuro, se podrían proponer
  mejoras, aventurarse con nuevos problemas y desarrollar técnicas más
  complejas para alcanzar objetivos todavía más ambiciosos.
\end{itemize}
