\chapter{Summary}
\section{Issue}
This project araises from the needed of improve and make automatic the
study of images within the area of the medicine, with the goal of 
increasing the diagnosis precision and accelerate it. Sometimes, the
machines which takes this images lacks of important characteristics 
that the doctors may need or facilities to monitoring the patient evolution.

In this project the object of study are the \gls{tco} images to monitoring
two diseases: \emph{\gls{uveitis}} and \emph{\gls{glaucoma}}. Due to that
it has received the collaboration of ophthalmologists who work at
\emph{Hospital Universitario 12 de Octubre}.

\section{Goals}
The medical goals are so clear to the team of ophthalmologists which
we collaborate. So much so that they had no problem to manually indicate on an 
image the information they would need to identify from the programs that 
manage the \gls{tco}. With this requirements, the next goals were defined 
to the project:
\begin{itemize}
\item Obtain images that ajust to the manually indicated one.
\item Make automatic the process or, at least, make it easier with
the purpose of reach it as less number of steps as possible.
\item Provide the knowledge get from the programs, information and 
dimensions in order to improve the understanding of the \gls{tco} and
the diseases.
\item Break the dependency from the software of the \gls{tco} machine
which can not be improved or adapted, due to author rights and patents, 
in order to resolve the problems the ophthalmologists have.
\item Learn how to get along in an external area to the computer science
with a great number of unknown techniques, trying to reach usefull 
results which can be applied in medicine.
\item Understand the problem as one, the goals and all the proposal ideas.
This way, in a future, it will be possible to suggest improvements, venture
on new issues and develop more complex techniques to reach so much 
ambicious goals.
\end{itemize}

