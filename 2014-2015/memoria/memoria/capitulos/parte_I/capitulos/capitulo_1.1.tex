\chapter{Introducción}
\section{Problema}
Este proyecto nace de la necesidad que tienen los médicos de mejorar y
automatizar el estudio de las imágenes obtenidas de los pacientes. Ya
sea porque las máquinas que generan dichas imágenes carecen de las
técnicas requeridas por los médicos o por el intento de automatizar el
seguimiento rutinario de la evolución de una enfermedad a partir de su
historial de imágenes en el tiempo. \\
Por ser el problema tan amplio y poco concreto, nos hemos centrado en
colaborar con oftalmólogos del \emph{Hospital Universitario 12 de
  Octubre} en el estudio imágenes de \gls{tcoa} tomadas para seguir la
evolución dos enfermedades, la \emph{uveitis} y el \emph{glaucoma}

\section{Objetivos}
Aunque el problema a primera vista es demasiado amplio e inabarcable
los objetivos están muy claros por los oftalmólogos. De hecho, les
pedimos que nos marcaran a mano lo que necesitaban estudiar de las
\gls{tcoa}. Así conseguimos enumerar los siguientes objetivos:
\begin{itemize}
\item Obtener imágenes tan fieles como sea posible a las marcadas
  manualmente.
\item Conseguir automatizar al máximo dicho proceso o al menos
  facilitarlo para poderse realizar en el menor número de pasos posibles.
\item Proporcionar los conocimientos obtenidos de los programas, datos,
  medidas\ldots\ para mejorar el entendimiento tanto de las \gls{tcoa}
  como de las enfermedades estudiadas.
\item Aprender a desenvolvernos en un entorno completamente ajeno al
  nuestro con un montón de técnicas para nosotros desconocidas para alcanzar
  resultados aplicables y útiles en el  mundo real y más concretamente
  en entornos hospitalarios.
\item Comprender en su totalidad el problema, los objetivos y todas
  las soluciones propuestas para en un futuro proponer mejoras,
  atrevernos con nuevos problemas y desarrollar técnicas más complejas
  para proponernos objetivos más difíciles.
\end{itemize}
