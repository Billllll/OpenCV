\chapter{Técnicas auxiliares}
\section{Origen}
En los siguientes capítulos se habla de usar técnicas diferentes
para cada paso, con parámetros definidos y concretos. \\
La elección de estas técnicas junto con sus parámetros no se trata de
una elección arbitraria. Para llegar a la mejor solución, es decir, la
que encuentre los datos necesarios de forma exacta y sea genérica (que
sirva para todas las imágenes), se llevó a cabo un estudio del
funcionamiento y resultados de todas las posibles técnicas a aplicar.
Para ello se desarrollaron, en la mayoría de casos, algoritmos
auxiliares de visión computarizada que permitieran encontrar de forma
eficaz la técnica más adecuada en cada caso.

\section{Implementación}
Dado que el número de técnicas es muy amplio, así como los valores que
pueden tomar sus parámetros, se desarrollaron una serie de programas
destinados a probar las transformaciones aplicadas por dichas técnicas
de forma fácil. Para ello se generaron diversas interfaces gráficas
que, valiéndose de barras, permiten ver en tiempo real las
transformaciones aplicadas a una imagen para así poder compararlas y
encontrar la que mejores resultados proporcione. Además este tipo de
interfaces gráficas permitieron a los oftalmólogos interactuar y
comprender los resultados obtenidos, así como también de hacer
propuestas e indicaciones para realizar mejoras que de otra manera no
podrían haber hecho.
