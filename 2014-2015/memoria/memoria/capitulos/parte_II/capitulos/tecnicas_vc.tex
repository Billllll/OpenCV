\chapter{Técnicas de visión computerizada}
Aquí se detallan todas las técnicas con ejemplos aplicadas o
descartadas a lo largo de la investigación.
\section{Operaciones básicas}
\subsection{Acceso a píxeles}
El acceso a los píxeles de una imagen se realiza a través de
\emph{Numpy} como una matriz de dimensiones \emph{y}, \emph{x}.

\subsection{Propiedades}
Las dos propiedades más utilizadas las imágenes son su altura y anchura.

\subsection{Región de interés}
Una región de interés de una imagen es la zona en la que estamos
interesados en procesar.

\section{Operaciones aritméticas}
\subsection{Superposición}
La superposición de dos imágenes (del mismo tipo, profundidad o una
que sea un valor escalar) se produce al sumarlas píxel a píxel
como dos matrices de \emph{Numpy} que al hacerlo con diferentes
intensidades da la sensación de superposición y transparencia. Dicha función viene dada por:
\begin{equation*}
g(x) = (1 - \alpha)f_0(x) + \alpha f_1(x)
\end{equation*}

\section{Cambios de color}

\section{Operaciones de umbral}
\subsection{Simples}
\subsection{Adaptativos}
\subsection{Binarización Otsu}

\section{Transformaciones geométricas}
\subsection{Rotación}

\section{Suavizado}
\subsection{Promedio}
\subsection{Gaussiano}
\subsection{Mediana}

\section{Transformaciones morfológicas}
\subsection{Erosión}
\subsection{Dilatación}
\subsection{Apertura}
\subsection{Cierre}
\subsection{Borde morfológico}

\section{Bordes}
\subsection{Sobel}
\subsection{Scharr}
\subsection{Laplaciana}

\section{Algoritmo Canny}

\section{Contornos}

\section{Histogramas}
\subsection{Ecualización}

\section{Transformada de Hough}
\subsection{Rectas}