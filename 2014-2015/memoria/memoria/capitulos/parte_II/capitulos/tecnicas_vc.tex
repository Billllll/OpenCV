\chapter{Técnicas de visión computerizada}
Aquí se detallan todas las técnicas aplicadas o descartadas 
a lo largo de la investigación. Se incluyen ejemplos.
\section{Operaciones básicas}
\subsection{Acceso a píxeles}
El acceso a los píxeles de una imagen se realiza a través de
\emph{Numpy} como una matriz de dimensiones [\emph{y}, \emph{x}],
siendo \emph{y} la altura y \emph{x} el ancho.
\begin{minted}{Python}
pixel = imagen[y, x]
\end{minted}

\subsection{Propiedades}
Las dos propiedades más utilizadas de las imágenes son su altura y anchura.
Se pueden obtener con la función \emph{shape}.
\begin{minted}{Python}
y, x = imagen.shape()
\end{minted}

\subsection{Región de interés}
La región de interés (ROI) de una imagen es la zona en la que estamos
interesados en procesar.

\section{Operaciones aritméticas}
\subsection{Superposición}
La superposición de dos imágenes (del mismo tipo, profundidad o una
que sea un valor escalar) es la suma matricial de sus píxeles,
dándole a cada imagen un peso diferente para dar la sensación
de superposición y transparencia. La función sería:
\begin{equation*}
g(n) = (1 - \alpha)f_0(n) + \alpha f_1(n)
\end{equation*}
Donde \emph{n} representa un punto de la imagen, \emph{\alpha} es 
una constante entre 0 y 1, y \emph{f_0} y \emph{f_1} las imágenes.
Expresado en forma de coordenadas:
\begin{equation*}
g(x, y) = (1 - \alpha)f_0(x, y) + \alpha f_1(x, y)
\end{equation*}
Donde \emph{x} e \emph{y} representan las coordenadas (horizontal y 
vertical respectivamente).


\section{Cambio de espacio de color}
Un espacio de color es un modelo matemático abstracto que describe 
la forma en la que los colores pueden representarse como tuplas de 
números. RGB, HSV o escala de grises son ejemplos de espacios de color. \\
El espacio de color de una imagen se puede cambiar con la función \emph{cvtColor}.
Esta función permite las siguientes conversiones:
\begin{itemize}
\item De \textbf{RGB} a \textbf{escala de grises}: CV_RGB2GRAY, CV_GRAY2RGB, CV_BGR2GRAY 
  o CV_GRAY2BGR.
\item De \textbf{RGB} a \textbf{CIE XYZ}: CV_BGR2XYZ, CV_RGB2XYZ, CV_XYZ2BGR o CV_XYZ2RGB.
\item De \textbf{RGB} a \textbf{YCrCb JPEG} (o YCC): CV_BGR2YCrCb, CV_RGB2YCrCb, CV_YCrCb2BGR 
  o CV_YCrCb2RGB.
\item De \textbf{RGB} a \textbf{HSV}: CV_BGR2HSV, CV_RGB2HSV, CV_HSV2BGR o CV_HSV2RGB.
\item De \textbf{RGB} a \textbf{HLS}: CV_BGR2HLS, CV_RGB2HLS, CV_HLS2BGR o CV_HLS2RGB.
\item De \textbf{RGB} a \textbf{Bayer}: CV_BayerBG2BGR, CV_BayerGB2BGR, CV_BayerRG2BGR, 
  CV_BayerGR2BGR, CV_BayerBG2RGB, CV_BayerGB2RGB, CV_BayerRG2RGB o CV_BayerGR2RGB.
\end{itemize}
Estas transformaciones permiten la detección rápida y sencilla de características
de interés.


\section{Operaciones de \emph{threshold}}
\subsection{Simples}
\subsection{Adaptativos}
\subsection{Binarización Otsu}

\section{Transformaciones geométricas}
\subsection{Rotación}

\section{Suavizado}
\subsection{Promedio}
\subsection{Gaussiano}
\subsection{Mediana}

\section{Transformaciones morfológicas}
\subsection{\emph{Erode}}
\subsection{\emph{Dilate}}
\subsection{Apertura}
\subsection{Cierre}
\subsection{Borde morfológico}

\section{Bordes}
\subsection{Sobel}
\subsection{Scharr}
\subsection{Laplaciana}

\section{Algoritmo \emph{Canny}}

\section{Contornos}

\section{Histogramas}
\subsection{Ecualización}

\section{Transformada de Hough}
\subsection{Rectas}