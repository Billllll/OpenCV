\chapter{Bibliotecas}
Todas las bibliotecas usadas son de código abierto, así
como también todo el código desarrollado con ellas.

\section{Bibliotecas de terceros}
\subsection{Numpy}
Biblioteca principal para cálculo científico con \emph{Python}. Destaca por:
\begin{itemize}
\item Matrices N-dimensionales para albergar las imágenes.
\item Operaciones de álgebra lineal para las matrices.
\item Integración con el resto de bibliotecas del proyecto.
\end{itemize}

\subsection{matplotlib}
Biblioteca para dibujar en 2D en \emph{Python}. Muy usada en el campo de 
visión computerizada para la visualización de histogramas.

\subsection{OpenCV}
Biblioteca de visión computerizada escrita en C++ y diseñada para
permitir su utilización en aplicaciones de tiempo real. Proporciona:
\begin{itemize}
\item Integración precisa con \emph{Python} y \emph{Numpy}.
\item Contiene funciones y algoritmos aplicables a todas las áreas de
  visión computerizada, especialmente para imágenes en 2D.
\item Creación de interfaces gráficas con paneles para
  facilitar la interacción con las técnicas aplicadas a las imágenes.
\end{itemize}

\subsection{SimpleCV}
Biblioteca de visión computerizada escrita en \emph{Python} sobre
\emph{OpenCV} que provee varios algoritmos complejos con 
una interfaz mucho más simple que OpenCV\@.

\section{Bibliotecas propias}

Bibliotecas desarrolladas por y para el proyecto. Son envolturas a
otras bibliotecas con el objetivo de adaptarlas a las necesidades
surgidas con las \gls{tcoa}.

\subsection{Envoltura del módulo argparse}
Biblioteca escrita para facilitar la integración con el módulo
\emph{argparse} de la biblioteca estándar de \emph{Python} e incorporar
procesamiento de argumentos por consola como:
\begin{itemize}
\item Modo depuración.
\item Procesamiento de imágenes individuales.
\item Procesamiento de carpetas de imágenes.
\end{itemize}

\subsection{Extracción de zonas de interés}
Biblioteca que se encarga de extraer la región que se va a estudiar
de una \gls{tcoa}.

\subsection{Corrección de la inclinación}
Biblioteca que corrige la inclinación de las \gls{tcoa}, poniéndolas
horizontales si no lo están.