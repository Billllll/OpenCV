\chapter{Técnicas auxiliares}
\section{Origen}
En los anteriores capítulos se habla de usar técnicas diferentes 
para cada paso, con parámetros definidos y concretos. \\
La elección de estas técnicas junto con sus parámetros no se trata
de una elección arbitraria. Para llegar a la mejor solución, es decir,
la que encuentre los datos necesarios de forma exacta y sea genérica
(que sirva para todas las imágenes), se llevó a cabo un estudio del
funcionamiento y resultados de todas las posibles técnicas a aplicar.
Para ello se desarrollaron, en la mayoría de casos, algoritmos 
auxiliares de visión computerizada que permitieran encontrar de 
forma eficaz la técnica más adecuada en cada caso.

\section{Implementación}
Dado que el número de técnicas era muy amplio, así como los valores
que pueden tomar sus parámetros, se desarrollaron una serie de 
programas destinados a testear las transformaciones aplicadas por
dichas técnicas de forma fácil. Para ello se generaron diversas
interfaces gráficas que, valiéndose de barras, permiten ver
en tiempo real las transformaciones aplicadas a una imagen para así
poder compararlas y encontrar la que mejores resultados proporcione.