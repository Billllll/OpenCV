\chapter{Conclusión}
\section{Conclusiones generales}
El uso de algoritmos de Visión Computerizada para tratamiento de
imágenes es aún algo novedoso y poco extendido. Sin embargo, su
potencial es altísimo. Incluso con unos conocimientos previos
casi inexistentes, en este proyecto ha quedado claro la gran
versatilidad de estas técnicas aplicadas a casos concretos del
mundo real. Aún queda mucho por investigar en este campo de la
informática, y es para nosotros un orgullo haber tenido la 
posibilidad de participar en un trabajo de estas características.\\
Cuando se habla de Visión Computerizada quizá se tienda a pensar,
inocentemente, que es un ámbito específico de la Informática. Sin
embargo va mucho más allá, pues sus funcionalidades son aplicables
a cualquier tipo de imagen, como en este caso lo han sido las 
\emph{tomografías}. Los resultados obtenidos en su tratamiento han
sido más que satisfactorios, tanto para nosotros como para el 
personal médico que ha colaborado en el proyecto, proporcionando
conocimiento técnico del área e imágenes de estudio.

\section{Conclusiones sobre la detección de poros en la papila}

\section{Conclusiones sobre la medición de la coroides}
En cuanto al uso de algoritmos de Visión Computerizada para la 
medición del espesor de la \emph{coroides} los resultados no
podrían ser mejores: se ha alcanzado una precisión de la medición
exacta para la gran mayoría de las imágenes y diferencias 
imperceptibles en los peores casos, cuando el ruido o la calidad
de la imagen complican la detección de los puntos.