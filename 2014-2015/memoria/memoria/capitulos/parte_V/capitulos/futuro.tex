\chapter{Futuro}
Como ya se profundizó en
el~\autoref{ch:estado_arte},~\nameref{ch:estado_arte}, la trayectoria
de la \emph{Visión computarizada} junto a las \emph{\gls{tco}} y al
incipiente aumento año tras año de investigaciones, estudios y
artículos dedicados, hace pensar en un futuro más que
prometedor. Incluso a pesar de los buenos resultados obtenidos en este
trabajo, existe un gran margen de mejora mediente el estudio y
aplicación de técnicas que por falta de tiempo y/o conocimientos se
tuvieron que descartar. Además, este trabajo sienta la base para otras
muchas ideas propuesta a largo plazo.\\
Por todo esto, se ha propuesto el siguiente plan de trabajo a
realizar.

\section{OpenCV 3}
A la par que se crean y avanzan técnicas y algoritmos de \emph{Visión
  computarizada} a un ritmo vertiginoso, también avanzan sus
implementaciones software de manera conjunta. En nuestro caso,
\emph{OpenCV}, la biblioteca estándar de la industria en \emph{Visión
  computarizada}. Cada vez es más funcional, rápida, y hace cada vez
más énfasis en la aceleración sobre distintos tipos de hardware
como demuestra la nueva versión
$3.0$\footnote{\url{http://opencv.org/opencv-3-0.html}} publicada a
principios del mes de la entrega de este proyecto.

\section{Aprendizaje automático}
Los algoritmos desarrollados durante este trabajo carecen de cualquier
técnica de \emph{aprendizaje automático}. Como futuro próximo, es de
obligado cumplimiento investigar y explotar este tipo de técnicas de
\emph{inteligencia artificial}, además de cualquier otra que se pudiera
aplicar del área abarcada por la \emph{inteligencia
  artificial}. Primero, un estudio preliminar sobre la viabilidad y
aplicación de \emph{aprendizaje supervisado} para posteriormente
aplicar \emph{aprendizaje no supervisado}. Con esto, se conseguiría
entrenar al algoritmo para procesar imágenes con excesos de ruidos y
abrir la puerta al estudio de estructuras oculares de difícil enfoque
por la máquina de \gls{tco}, así como detectar posibles indicios de
enfermedades en una etapa temprana.

\section{Nuevos proyectos}
Con el fin de este trabajo comienza a cobrar sentido plantearse otros
nuevos proyectos e ideas con más capacidad de alcance y profundidad en
materia de \emph{Visión computarizada}, tanto a imágenes \gls{tco} como
a otro tipo de imágenes y aplicaciones reales a las que se pueda extender.
