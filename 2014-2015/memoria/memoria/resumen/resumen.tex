\section*{Resumen}
Durante la realización de un diagnóstico médico se genera una gran
cantidad de información. En muchas ocasiones esta información cobra
forma de imágenes \emph{\citep*[1. The Analysis of Medical Images,
  2. Digital Image Acquisition]{toennies2012guide}}: radiografías,
tomografías, resonancias magnéticas, ecografías, etc. Esto crea la
necesidad de mejorar las técnicas de estudio de estas imágenes
\emph{\citep*[4. Image Enhancement]{toennies2012guide}}, facilitando y
automatizando su interpretación, de forma que el diagnóstico sea más
rápido y exacto. Así se pretende aumentar la precisión en la detección
y el seguimiento de enfermedades.\\
Este proyecto se centra en el estudio de un tipo de imagen concreto:
las tomografías generadas por una máquina de Tomografía de Coherencia
Óptica. Esta máquina mediante la reflexión de ondas de luz crea una
representación visual de los tejidos de la parte interna del globo
ocular (para más detalle consultar el primer capítulo de la parte
III). Más exactamente, el estudio llevado a cabo se centra en
tomografías de la \gls{papila-optica} o cabeza del nervio óptico y la
\gls{coroides}:
\begin{itemize}
\item Sobre la \gls{papila-optica} se necesita definir, marcar y medir
  el tamaño de los poros de la \emph{\gls{lamina-cribosa}}, que están
  siendo objeto de estudio debido a su posible relación con la
  aparición del \gls{glaucoma}.
\item Sobre la \gls{coroides} se necesita definir, marcar y medir el
  grosor de la misma debido a la relación existente entre su grosor y
  diversas enfermedades, entre ellas, la \emph{\gls{uveitis}}.
\end{itemize}
El objetivo es hacer un tratamiento lo más automatizado posible de
estas imágenes utilizando algoritmos de \emph{Visión Computarizada}
mediante su implementación con distintas bibliotecas de \emph{software
  libre} tanto propias como de terceros que se explicarán más
adelante. La razón de crear y usar exclusivamente \emph{software
  libre} surge, por una parte, de no depender del software costosísimo
patentado y protegido de las máquinas, así como de la necesidad de
conocimiento profundo y transparente de su funcionamiento, debido a la
aparición de nuevas necesidades muy precisas de los oftalmólogos,
tanto en las tareas más rutinarias como en las más pioneras. Estas
necesidades siempre estarán un paso por delante de las facilidades y
adaptaciones proporcionadas por las actualizaciones propietarias de
las grandes compañías, siendo éstas además difícilmente costeables.

\newpage

\section*{Abstract}
-During the execution of medical diagnosis a lot of information is generated.
-Many times, this information takes the form of images
-\emph{\citep*[1. The Analysis of Medical Images, 2. Digital Image
-  Acquisition]{toennies2012guide}}: radiographies, tomographies,
-magnetic resonances, ecographies, etc.
-This creates the need to improve the study techniques which are used
-for these images \emph{\citep*[4. Image Enhancement]{toennies2012guide}},
-making their interpretation easier and automatic in order to
-get a fast and exact diagnosis.Being the objetive of this increasing the 
precission while detecting and monitoring the illness.\\
This project focuses on the study of a particular type of image: those
generated by optical coherence tomography machines, which creates a
visual representation of the inner leyers of the eye by the reflection
of light waves (for more information consult the first chapter of Part
III). More precisely, the study objective is focused on scans of the
optic disc or optic nerve head and the choroid:
\begin{itemize}
\item On the optic nerve it is necessary to define, mark and measure
  the size of the pores of the \emph{lamina cribrosa}, which are being
  studied because of their potential relation to the onset of \emph{glaucoma}.
\item On the choroid it is necessary to define, mark and measure its
  thickness because of the relation between it and various diseases, including uveitis.
\end{itemize}
The goal is to make the treatment of the images as automatic as
possible using Computer Vision algorithms through its implementation
with various free software libraries. Some of the libraries that have
been used were self-implemented and others made by third parts as
free-software libraries. These libraries will be detailed later. The
reason for creating and using free software exclusively comes, on the
one hand, from not depending on the very expensive patented and protected
software and, on the other hand, the need for thorough transparent
knowledge of its operation. This will respond to the new needs of the
ophthalmologists in the routine tasks and in the pioneer ones. These needs 
are always one step ahead of the
facilities and adaptations proprietary updates provided by big
companies, which are also hardly affordable.