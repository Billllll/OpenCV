\section*{Resumen}
Durante la realización de un diagnóstico médico se genera una gran
cantidad de información. En muchas ocasiones esta información cobra
forma de imágenes: radiografías, tomografías, resonancias magnéticas, 
ecografías, etc. Esto crea la necesidad de mejorar las técnicas
de estudio de estas imágenes, facilitando y automatizando su interpretación,
de forma que el diagnóstico sea más rápido y exacto. Así se pretende
aumentar la precisión en la detección y el seguimiento de enfermedades.\\
Este proyecto trata un tipo de imagen concreto: las tomografías generadas 
por una \gls{tcoa}. Una \gls{tcoa} es una máquina que mediante la reflexión 
de ondas de luz crea una representación visual de los tejidos de la parte 
interna del ojo (Para más detalle consultar la sección correspondiente 
a \gls{tcoa}). Más exactamente, el estudio llevado a cabo se centra en
tomografías de la papila retiniana y la coroides:
\begin{itemize}
\item Sobre la papila retiniana se necesita definir, marcar y medir el tamaño
  de porosidades, que están siendo objeto de estudio por parte de personal
  médico debido a su posible relación con la aparición del glaucoma.
\item Sobre la coroides se necesita definir, marcar y medir el grosor de 
  la misma debido a la relación existente entre su inflamación y la
  aparición de la uveitis.
\end{itemize}
Para ello se han utilizado algoritmos de \emph{Visión Computerizada} 
mediante su implementación en OpenCV que se explicarán más adelante.
\section*{Abstract}

