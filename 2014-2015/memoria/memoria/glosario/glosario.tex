% http://tex.stackexchange.com/a/8951
% http://osl.ugr.es/CTAN/macros/latex/contrib/glossaries/glossariesbegin.pdf

% TCO acrónimo y glosario
\newdualentry{tco} % etiqueta
  {TCO}            % abreviatura
  {Tomografía de Coherencia Óptica}  % forma larga
  {Técnica de imagen no invasiva que utiliza ondas de luz coherente
    para obtener imágenes de los tejidos vivos} % descripción

% ECMI acrónimo y glosario
\newdualentry{ECMI} % etiqueta
  {ECMI}            % abreviatura
  {European Consortium for Mathematics in Industry}  % forma larga
  {Consorcio de instituciones académicas y compañías industriales con
    los objetivos de educar, promover y fomentar expertos en el uso de
    modelos matemáticos en cualquier actividad social o de importancia
    económica a nivel europeo} % descripción

% ORB acrónimo y glosario
  \newdualentry{ORB} % etiqueta
  {ORB} % abreviatura
  {Oriented FAST and Rotated BRIEF} % forma larga
  {Algorimo para el reconocimiento de estructuras o puntos de interés
    resistente al ruido. Basado en descriptores visuales del algoritmo
    BRIEF (Binary Robust Independent Elementary Features) y en el
    detector de puntos claves del algoritmo FAST (Features from
    Accelerated Segment Test). Mucho más eficiente que ambos y libre
    de patentes} % descripción

\newglossaryentry{coroides}
{
  name={coroides},
  description={Parte de la úvea situada entre la retina y la esclera,
    encargada de la nutrición y oxigenación de la retina externa}
}

\newglossaryentry{papila-optica}
{
  name={papila óptica},
  description={Papila óptica, disco óptico o cabeza del nervio óptico
    es una zona circular situada en la parte posterior del globo
    ocular, por donde salen del ojo los axones de las células
    ganglionares de la retina que forman el nervio óptico. La zona
    externa, ocupada por los axones, es el anillo neural, y la zona
    central, en la que no hay axones, se denomina excavación, y está
    aumentada en el glaucoma}
}

\newglossaryentry{glaucoma}
{
  name={glaucoma},
  description={Neuropatía óptica caracterizada por la muerte de las
    células ganglionares de la retina, y que se manifiesta
    clínicamente por el aumento de la excavación papilar (por la
    pérdida de los axones que forman el anillo neural de la papila) y
    por defectos típicos en el campo visual. Si no se trata, produce
    ceguera irreversible. El aumento de la presión intraocular es el
    principal factor de riesgo para su aparición}
}

\newglossaryentry{uvea}
{
  name={úvea},
  description={Capa vascular del globo ocular, que mantiene la
    temperatura y proporciona nutrición y oxigenación a diversas
    estructuras intraoculares. Está formada por el iris, en la zona
    anterior del globo ocular; el cuerpo ciliar, que proporciona
    anclaje al cristalino e interviene en la acomodación y producción
    de humor acuoso; y la coroides, situada en la parte posterior del
    ojo, que proporciona nutrición a la retina externa}
}

\newglossaryentry{uveitis}
{
  name={uveítis},
  description={Conjunto de enfermedades en las que existe inflamación de la úvea}
}

\newglossaryentry{fovea}
{
  name={fóvea},
  description={Continuación de la esclera por la que pasan los haces
    de fibras nerviosas. Está constituida por tejido conectivo
    perforado y revestido por astrocitos. La lámina cribosa divide la
    cabeza del nervio óptico en tres partes: prelaminar,
    correspondiente a la zona visible con el oftalmoscopio; laminar,
    que es la lámina cribosa propiamente dicha, visible en ocasiones con
    el oftalmoscopio cuando la excavación es profunda y grande; y
    retrolaminar, la zona más posterior que no es visible}
}

\newglossaryentry{mBruch}
{
  name={membrana de Bruch},
  description={Capa más interna de la coroides. Está en contacto con los
    segmentos externos de los fotoreceptores}
}

\newglossaryentry{lamina-cribosa}
{
  name={lámina cribosa},
  description={Continuación de la esclera por la que pasan los haces
    de fibras nerviosas. Está constituida por tejido conectivo
    perforado y revestido por astrocitos. La lámina cribosa divide la
    cabeza del nervio óptico en tres partes: prelaminar,
    correspondiente a la zona visible con el oftalmoscopio; laminar,
    que es la lámina cribosa propiamente dicha, visible en ocasiones con
    el oftalmoscopio cuando la excavación es profunda y grande; y
    retrolaminar, la zona más posterior que no es visible}
}

\newglossaryentry{esclera}
{
  name={esclera},
  description={La esclera o esclerótica es una membrana de color de 
    color blanco que constituye la parte más externa del globo ocular.
    Cumple la función de proteger a los elementos internos. También
    cubre la coroides por detrás}
}